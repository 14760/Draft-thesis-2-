\chapter{Phenotypes and the Post synaptic proteome network}
\section{Phenotypes considered}


\section{Intellectual disabilitity}
Intellectual disability and the genetics of intelligence. 
\section{Degree and id genes}
Examining the centrality measures for the 692 ID genes from cite(vissers2016genetic) implicated in intellectual disability,  245 are in the PSP.


 The median degree and mean degree are higher but not significantly so (Two sample Wilcoxon rank sum test with continuity correction - Mann-Whitney, W = 454524, p-value = 0.055 calculated in R) although the difference in the betweenness centrality is significant (Wilcoxon rank sum test with continuity correction. W = 467100, p-value = 0.006944, two sided for visser greater than W = 467100, p-value = 0.0032 ) see table \ref{table:degree_visser}, \ref{table:visser_betweenness}. \todo{90 centile degree for different diseases}
 
692 genes were identified from Visser et al 2016 Nature Review Genetics (ref) . The paper states there are ~700 genes and over 700 genes but I cannot identify a precise figure and the supplemental material is in PDF format and not machine readable.
692 genes were extracted using the python module pdf2text (ref) and confirming the list of genes gene identity in ToppGene (ref). These were stored as a pdf.
245 synaptic genes were identified. 35.4\% of the genes reported by Visser are synaptic and represent 7\% of the 3457 vertex PSP.  
There was no significant difference in degree distribution between ID genes and the PSP as a whole 
(median 10.0 Visser, 8 PSP, Mean 19.24 v 17.64 – Visser, p = 0.054 Wilcoxon test in R)
There was a significant difference in betweenness centrality (Median 317 v Median 540, p=0.006944 ) higher betweenness centrality. The genes appeared to be underrepresented as articulation points in the PSP (3 of 203)
Eigenvector similarity appears to be non significant

They are unevenly distributed with spectral clustering. The top group is group 4 with 9 of 31 genes associated. A high ranking GO term for group 4 is peroxisome related (8 genes). 7 of the 9 visser genes were associated with this GO term showing some modularity in function in at least a proportion of cases. Group 5 has 11 genes from Visser (9.2\% of community, slightly more than average of PSP).

Top groups spectral
\begin{table}[ht]
\centering
\begin{tabular}{clll}
\hline
 Spectral group &Freq visser& PSP group size &frac of group\\
              4&           9 &            31&    0.29\\
          47  &         6   &          27  &  0.222\\
          19   &        3    &         20   & 0.150\\
         45    &      15     &       124 &   0.121\\
           43     &     18      &      150  &  0.120\\
           22      &     5       &      42   & 0.119\\
         26       &   13      &      120 &   0.108\\
        28       &    4     &        38  &  0.105\\
            5      &    11    &        112   & 0.098\\
          24     &      7   &          76 &   0.092\\
\end{tabular}
\caption{Distribution of ID genes from Visser over spectral groups}
\label{table:Distribution of ID genesw from Visser over spectral groups}
\end{table}
  %from \hyperref{/home/grant/Dropbox/stront_share/PhD_bits/id}


% latex table generated in R 3.6.1 by xtable 1.8-4 package
% Mon Oct  7 13:54:07 2019
\begin{table}[ht]
\centering
\begin{tabular}{rrrrrrrr}
  \hline
 & Min. & 1st Qu. & Median & Mean & 3rd Qu. & Max. & n\\ 
  \hline
all\_synapse & 1.00 & 4.00 & 8.00 & 17.64 & 19.00 & 535.00 &3457\\ 
  visser synaptic & 1.00 & 4.00 & 10.00 & 19.24 & 22.00 & 263.00 & 245\\ 
   \hline
\end{tabular}
\caption{Difference in degree. PSP genes in cite(vissers2016genetic) p = 0.54}
\label{table:degree_visser}
\end{table}

% latex table generated in R 3.6.1 by xtable 1.8-4 package
% Mon Oct  7 14:08:05 2019
\begin{table}[ht]
\centering
\begin{tabular}{rrrrrrrr}
  \hline
 & Min. & 1st Qu. & Median & Mean & 3rd Qu. & Max. & n\\ 
  \hline


All PSP & 0.00 & 43.29 & 317.04 & 3421.08 & 1571.57 & 644670.69 & 3457 \\ 
  Visser genes & 0.00 & 68.94 & 540.62 & 3485.36 & 2761.75 & 85706.61 & 245\\ 
   \hline
\end{tabular}
\caption{Difference in betweenness centrality. PSP genes in citep(vissers2016genetic) p=0.0069}
\label{table:visser_betweenness}
\end{table}
cite(zabaneh2018genome)
\subsubsection{Essential genes and degree}
Lek
Database of essential genes. 

