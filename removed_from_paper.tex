\chapter{removed from paper}

\section{title and authors}
Spectral clustering reveals a role for the glutamatergic receptor complex in human cognitive abilities

Robertson, G, Hill W D, McLean C, Sorokina O, [Heil, Emilia?],[other CCACE] [EA2
people who got the data without UKBB],Simpson IT, Deary IJ, Armstrong JD



\section{abstract}
Abstract

Abstract: 150-250 words 
142 words
20.09 187 words
Introduction: 1,500 words
2039 at present
1634 at present 11.09
1641 24.09
Article: 3,500 words, excluding abstract and references.
5846 words at present
4730 at present 11.09
4572 words.


Max of 5 Figures and tables –
 2 figures and 2 or 3 tables depending on version chosen
19 supplemental tables
Max of 75 References – 70 ref just now, 69 just now

Abstract (187 words)
The proteins that interact in the region of the neuron’s post synaptic density, the post synaptic proteome (PSP), have a modular structure and their function is profoundly affected by the pattern of their interactions.
Previous studies have shown an enriched association between genetic variations in synaptic genes and individual differences in measured intelligence. We describe, using community detection methods, meso-scale structures in the PSP we can divide the PSP into substructures based on molecular interactions. Using competitive gene set analysis in MAGMA, we investigate whether any of these substructures (communities) have an enriched association with intelligence or the related phenotype of educational attainment. We use genetic data from UK Biobank (n=120 934 intelligence, n=273 274 educational attainment) to show that a community closely associated with glutamate receptors has an enriched association with intelligence and educational attainment (p=0.002 Education, p=0.014 Intelligence). We replicate these findings using similar but non-overlapping cohorts (Education n=217 569, p=0.0084; Intelligence n=78 308, p=0.015, alpha Bonferroni 0.045). We find that the source of the enrichment signal is not the glutamate receptors themselves but their interaction neighbours, a finding which would not be apparent using standard manually curated approaches gene-sets (e.g. GO, pathways) to generating gene sets for (GSA).

  


\section{From centrality}
\subsection{Presynaptic proteome}



\section{Disease modules}
\section{GSA}
\subsection{rnk files}
Link behind comment \url{http://software.broadinstitute.org/cancer/software/gsea/wiki/index.php/Data_formats#RNK:_Ranked_list_file_format_.28.2A.rnk.29}
\clearpage{}
